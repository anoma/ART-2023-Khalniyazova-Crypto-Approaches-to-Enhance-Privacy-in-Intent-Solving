% mscguide.tex
% v1.0, released 11 Nov 2019
% Copyright 2019 Cambridge University Press


\usepackage[english]{babel}
\usepackage[T1]{fontenc}
\usepackage[utf8]{inputenc}
\usepackage{amsmath}
\usepackage{utf8-symbols}
\usepackage{microtype}
\usepackage{nccmath}
\usepackage{textcomp}
\usepackage{mathtools}
\usepackage{xparse}
\usepackage{bbold}
\usepackage{stmaryrd}
\usepackage{wrapfig}
\usepackage{multirow}
\usepackage{fancyvrb}
\usepackage{xcolor}
\usepackage{float}
\usepackage{graphicx}
\usepackage{lineno}
\usepackage{dirtytalk}
\usepackage{epigraph}
\usepackage{tikz}
\usepackage{quiver}
\usepackage{minted}
\usepackage{amsmath}
\usepackage{amsthm}
\usepackage{amsfonts}
\usepackage{textcomp}
\usepackage{latexsym}
\allowdisplaybreaks

% correct spacing depending on the parameters.  In this case
% `all' exposes all the definitions of the package, and `british'
% makes sure that there is no comma after e.g. or i.e.
\usepackage[all,british]{foreign}

\newtheorem{example}[therm]{Example}
% \newtheorem{definition}[therm]{Definition}
% \newtheorem{lemma}[therm]{Lemma}

% \let\citep\cite
\usepackage{mathpartir}

% \usepackage[capitalize,noabbrev]{cleveref}
\usepackage{thmtools}

% The following let the reader click on the qed square
% to visit the Agda formalisation.
\newenvironment{linkproof}[2][]{%
    \begin{proof}[#1]
        \renewcommand{\qedsymbol}{\href{#2}{{ \(\qedsymbol\)}}}
    }{%
    \end{proof}
}
\newenvironment{linkdefinition}[2][]{%
    \def\linkagdadefinition{#2}
    \begin{definition}[#1]
}%
{%
    \hfill\href{\linkagdadefinition}{\(\diamond\)}
    \end{definition}
}


\numberwithin{equation}{section}
\usetikzlibrary{decorations.markings}
\usetikzlibrary{decorations.pathreplacing}
\usetikzlibrary{matrix}
\usetikzlibrary{arrows}
\usetikzlibrary{chains}
\usetikzlibrary{positioning}
\usetikzlibrary{scopes}
\usetikzlibrary{decorations.pathmorphing}
\graphicspath{{./images/}}
\input{macros}
\usepackage{cleveref}

\bibliographystyle{ACM-Reference-Format}
% \renewcommand{\baselinestretch}{2}

